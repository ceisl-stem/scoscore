% Options for packages loaded elsewhere
\PassOptionsToPackage{unicode}{hyperref}
\PassOptionsToPackage{hyphens}{url}
\PassOptionsToPackage{dvipsnames,svgnames,x11names}{xcolor}
%
\documentclass[
  letterpaper,
  DIV=11,
  numbers=noendperiod,
  oneside]{scrartcl}

\usepackage{amsmath,amssymb}
\usepackage{iftex}
\ifPDFTeX
  \usepackage[T1]{fontenc}
  \usepackage[utf8]{inputenc}
  \usepackage{textcomp} % provide euro and other symbols
\else % if luatex or xetex
  \usepackage{unicode-math}
  \defaultfontfeatures{Scale=MatchLowercase}
  \defaultfontfeatures[\rmfamily]{Ligatures=TeX,Scale=1}
\fi
\usepackage{lmodern}
\ifPDFTeX\else  
    % xetex/luatex font selection
  \setmainfont[]{spectral}
  \setsansfont[]{rubik}
  \setmonofont[]{JetBrains Mono}
\fi
% Use upquote if available, for straight quotes in verbatim environments
\IfFileExists{upquote.sty}{\usepackage{upquote}}{}
\IfFileExists{microtype.sty}{% use microtype if available
  \usepackage[]{microtype}
  \UseMicrotypeSet[protrusion]{basicmath} % disable protrusion for tt fonts
}{}
\makeatletter
\@ifundefined{KOMAClassName}{% if non-KOMA class
  \IfFileExists{parskip.sty}{%
    \usepackage{parskip}
  }{% else
    \setlength{\parindent}{0pt}
    \setlength{\parskip}{6pt plus 2pt minus 1pt}}
}{% if KOMA class
  \KOMAoptions{parskip=half}}
\makeatother
\usepackage{xcolor}
\usepackage[left=1in,marginparwidth=2.0666666666667in,textwidth=4.1333333333333in,marginparsep=0.3in]{geometry}
\setlength{\emergencystretch}{3em} % prevent overfull lines
\setcounter{secnumdepth}{-\maxdimen} % remove section numbering
% Make \paragraph and \subparagraph free-standing
\ifx\paragraph\undefined\else
  \let\oldparagraph\paragraph
  \renewcommand{\paragraph}[1]{\oldparagraph{#1}\mbox{}}
\fi
\ifx\subparagraph\undefined\else
  \let\oldsubparagraph\subparagraph
  \renewcommand{\subparagraph}[1]{\oldsubparagraph{#1}\mbox{}}
\fi

\usepackage{color}
\usepackage{fancyvrb}
\newcommand{\VerbBar}{|}
\newcommand{\VERB}{\Verb[commandchars=\\\{\}]}
\DefineVerbatimEnvironment{Highlighting}{Verbatim}{commandchars=\\\{\}}
% Add ',fontsize=\small' for more characters per line
\usepackage{framed}
\definecolor{shadecolor}{RGB}{254,254,254}
\newenvironment{Shaded}{\begin{snugshade}}{\end{snugshade}}
\newcommand{\AlertTok}[1]{\textcolor[rgb]{0.47,0.16,0.63}{#1}}
\newcommand{\AnnotationTok}[1]{\textcolor[rgb]{0.41,0.41,0.41}{#1}}
\newcommand{\AttributeTok}[1]{\textcolor[rgb]{0.65,0.35,0.00}{#1}}
\newcommand{\BaseNTok}[1]{\textcolor[rgb]{0.47,0.16,0.63}{#1}}
\newcommand{\BuiltInTok}[1]{\textcolor[rgb]{0.33,0.33,0.33}{#1}}
\newcommand{\CharTok}[1]{\textcolor[rgb]{0.00,0.50,0.00}{#1}}
\newcommand{\CommentTok}[1]{\textcolor[rgb]{0.41,0.41,0.41}{#1}}
\newcommand{\CommentVarTok}[1]{\textcolor[rgb]{0.41,0.41,0.41}{\textit{#1}}}
\newcommand{\ConstantTok}[1]{\textcolor[rgb]{0.85,0.12,0.09}{#1}}
\newcommand{\ControlFlowTok}[1]{\textcolor[rgb]{0.85,0.12,0.09}{#1}}
\newcommand{\DataTypeTok}[1]{\textcolor[rgb]{0.47,0.16,0.63}{#1}}
\newcommand{\DecValTok}[1]{\textcolor[rgb]{0.47,0.16,0.63}{#1}}
\newcommand{\DocumentationTok}[1]{\textcolor[rgb]{0.41,0.41,0.41}{\textit{#1}}}
\newcommand{\ErrorTok}[1]{\textcolor[rgb]{0.47,0.16,0.63}{#1}}
\newcommand{\ExtensionTok}[1]{\textcolor[rgb]{0.33,0.33,0.33}{#1}}
\newcommand{\FloatTok}[1]{\textcolor[rgb]{0.65,0.35,0.00}{#1}}
\newcommand{\FunctionTok}[1]{\textcolor[rgb]{0.02,0.16,0.49}{#1}}
\newcommand{\ImportTok}[1]{\textcolor[rgb]{0.33,0.33,0.33}{#1}}
\newcommand{\InformationTok}[1]{\textcolor[rgb]{0.41,0.41,0.41}{#1}}
\newcommand{\KeywordTok}[1]{\textcolor[rgb]{0.85,0.12,0.09}{#1}}
\newcommand{\NormalTok}[1]{\textcolor[rgb]{0.33,0.33,0.33}{#1}}
\newcommand{\OperatorTok}[1]{\textcolor[rgb]{0.00,0.46,0.62}{#1}}
\newcommand{\OtherTok}[1]{\textcolor[rgb]{0.85,0.12,0.09}{#1}}
\newcommand{\PreprocessorTok}[1]{\textcolor[rgb]{0.47,0.16,0.63}{#1}}
\newcommand{\RegionMarkerTok}[1]{\textcolor[rgb]{0.33,0.33,0.33}{#1}}
\newcommand{\SpecialCharTok}[1]{\textcolor[rgb]{0.00,0.46,0.62}{#1}}
\newcommand{\SpecialStringTok}[1]{\textcolor[rgb]{0.00,0.50,0.00}{#1}}
\newcommand{\StringTok}[1]{\textcolor[rgb]{0.00,0.50,0.00}{#1}}
\newcommand{\VariableTok}[1]{\textcolor[rgb]{0.65,0.35,0.00}{#1}}
\newcommand{\VerbatimStringTok}[1]{\textcolor[rgb]{0.00,0.50,0.00}{#1}}
\newcommand{\WarningTok}[1]{\textcolor[rgb]{0.41,0.41,0.41}{\textit{#1}}}

\providecommand{\tightlist}{%
  \setlength{\itemsep}{0pt}\setlength{\parskip}{0pt}}\usepackage{longtable,booktabs,array}
\usepackage{calc} % for calculating minipage widths
% Correct order of tables after \paragraph or \subparagraph
\usepackage{etoolbox}
\makeatletter
\patchcmd\longtable{\par}{\if@noskipsec\mbox{}\fi\par}{}{}
\makeatother
% Allow footnotes in longtable head/foot
\IfFileExists{footnotehyper.sty}{\usepackage{footnotehyper}}{\usepackage{footnote}}
\makesavenoteenv{longtable}
\usepackage{graphicx}
\makeatletter
\def\maxwidth{\ifdim\Gin@nat@width>\linewidth\linewidth\else\Gin@nat@width\fi}
\def\maxheight{\ifdim\Gin@nat@height>\textheight\textheight\else\Gin@nat@height\fi}
\makeatother
% Scale images if necessary, so that they will not overflow the page
% margins by default, and it is still possible to overwrite the defaults
% using explicit options in \includegraphics[width, height, ...]{}
\setkeys{Gin}{width=\maxwidth,height=\maxheight,keepaspectratio}
% Set default figure placement to htbp
\makeatletter
\def\fps@figure{htbp}
\makeatother

<script src="site_libs/htmlwidgets-1.6.2/htmlwidgets.js"></script>
<link href="site_libs/datatables-css-0.0.0/datatables-crosstalk.css" rel="stylesheet" />
<script src="site_libs/datatables-binding-0.28/datatables.js"></script>
<script src="site_libs/jquery-3.6.0/jquery-3.6.0.min.js"></script>
<link href="site_libs/dt-core-1.13.4/css/jquery.dataTables.min.css" rel="stylesheet" />
<link href="site_libs/dt-core-1.13.4/css/jquery.dataTables.extra.css" rel="stylesheet" />
<script src="site_libs/dt-core-1.13.4/js/jquery.dataTables.min.js"></script>
<link href="site_libs/crosstalk-1.2.0/css/crosstalk.min.css" rel="stylesheet" />
<script src="site_libs/crosstalk-1.2.0/js/crosstalk.min.js"></script>
<script src="site_libs/plotly-binding-4.10.2/plotly.js"></script>
<script src="site_libs/setprototypeof-0.1/setprototypeof.js"></script>
<script src="site_libs/typedarray-0.1/typedarray.min.js"></script>
<link href="site_libs/plotly-htmlwidgets-css-2.11.1/plotly-htmlwidgets.css" rel="stylesheet" />
<script src="site_libs/plotly-main-2.11.1/plotly-latest.min.js"></script>
\KOMAoption{captions}{tablesignature}
\makeatletter
\makeatother
\makeatletter
\makeatother
\makeatletter
\@ifpackageloaded{caption}{}{\usepackage{caption}}
\AtBeginDocument{%
\ifdefined\contentsname
  \renewcommand*\contentsname{Table of contents}
\else
  \newcommand\contentsname{Table of contents}
\fi
\ifdefined\listfigurename
  \renewcommand*\listfigurename{List of Figures}
\else
  \newcommand\listfigurename{List of Figures}
\fi
\ifdefined\listtablename
  \renewcommand*\listtablename{List of Tables}
\else
  \newcommand\listtablename{List of Tables}
\fi
\ifdefined\figurename
  \renewcommand*\figurename{Figure}
\else
  \newcommand\figurename{Figure}
\fi
\ifdefined\tablename
  \renewcommand*\tablename{Table}
\else
  \newcommand\tablename{Table}
\fi
}
\@ifpackageloaded{float}{}{\usepackage{float}}
\floatstyle{ruled}
\@ifundefined{c@chapter}{\newfloat{codelisting}{h}{lop}}{\newfloat{codelisting}{h}{lop}[chapter]}
\floatname{codelisting}{Listing}
\newcommand*\listoflistings{\listof{codelisting}{List of Listings}}
\makeatother
\makeatletter
\@ifpackageloaded{caption}{}{\usepackage{caption}}
\@ifpackageloaded{subcaption}{}{\usepackage{subcaption}}
\makeatother
\makeatletter
\@ifpackageloaded{tcolorbox}{}{\usepackage[skins,breakable]{tcolorbox}}
\makeatother
\makeatletter
\@ifundefined{shadecolor}{\definecolor{shadecolor}{rgb}{.97, .97, .97}}
\makeatother
\makeatletter
\makeatother
\makeatletter
\@ifpackageloaded{sidenotes}{}{\usepackage{sidenotes}}
\@ifpackageloaded{marginnote}{}{\usepackage{marginnote}}
\makeatother
\makeatletter
\makeatother
\makeatletter
\@ifpackageloaded{tikz}{}{\usepackage{tikz}}
\makeatother
        \newcommand*\circled[1]{\tikz[baseline=(char.base)]{
          \node[shape=circle,draw,inner sep=1pt] (char) {{\scriptsize#1}};}}  
                  
\ifLuaTeX
  \usepackage{selnolig}  % disable illegal ligatures
\fi
\IfFileExists{bookmark.sty}{\usepackage{bookmark}}{\usepackage{hyperref}}
\IfFileExists{xurl.sty}{\usepackage{xurl}}{} % add URL line breaks if available
\urlstyle{same} % disable monospaced font for URLs
\hypersetup{
  pdftitle={SCOscore: School Corporation Opportunity Score},
  pdfauthor={Jeremy Price},
  colorlinks=true,
  linkcolor={blue},
  filecolor={Maroon},
  citecolor={Blue},
  urlcolor={Blue},
  pdfcreator={LaTeX via pandoc}}

\title{SCOscore: School Corporation Opportunity Score}
\author{Jeremy Price}
\date{}

\begin{document}
\maketitle
\ifdefined\Shaded\renewenvironment{Shaded}{\begin{tcolorbox}[borderline west={3pt}{0pt}{shadecolor}, boxrule=0pt, interior hidden, frame hidden, enhanced, sharp corners, breakable]}{\end{tcolorbox}}\fi

\hypertarget{introduction}{%
\subsection{Introduction}\label{introduction}}

Quarto enables you to weave together content and executable code into a
finished document. To learn more about Quarto see
\url{https://quarto.org}.

\hypertarget{methods}{%
\subsection{Methods}\label{methods}}

When you click the \textbf{Render} button a document will be generated
that includes both content and the output of embedded code. You can
embed code like this:

\[
f(S_a^\prime) = \begin{cases}
(S_{IN} - S_a) + 1, & \text{if } S_a < S_{IN} \\
0, &\text{otherwise}
\end{cases}
\]

Then\ldots{}

\[
S_{ac} = \frac{\left(S_{a_{lit}}^\prime + S_{a_{math}}^\prime + S_{a_{gpc}}^\prime\right)}{3} + 1
\]

Finally\ldots{}

\[
S_{O} = \frac{\left[\left(P_{urm} \times 1.5\right) + \left(P_{frl} \times 1.5\right) + S_{ac}\right]}{3}
\] This is what it looks like in R:

\hypertarget{annotated-cell-1}{%
\label{annotated-cell-1}}%
\begin{Shaded}
\begin{Highlighting}[]
\NormalTok{school\_corp\_frame }\OtherTok{\textless{}{-}}\NormalTok{ school\_corp\_frame }\SpecialCharTok{|\textgreater{}}
  \FunctionTok{mutate}\NormalTok{(}
    \AttributeTok{adj\_3rd =} \FunctionTok{if\_else}\NormalTok{(}
\NormalTok{      ela\_3rd }\SpecialCharTok{\textless{}}\NormalTok{ state\_3rd\_proficiency, }\hspace*{\fill}\NormalTok{\circled{1}}
\NormalTok{      (state\_3rd\_proficiency }\SpecialCharTok{{-}}\NormalTok{ ela\_3rd }\SpecialCharTok{+} \DecValTok{1}\NormalTok{), }\hspace*{\fill}\NormalTok{\circled{2}}
      \DecValTok{0} \hspace*{\fill}\NormalTok{\circled{3}}
\NormalTok{    )}
\NormalTok{  ) }\SpecialCharTok{|\textgreater{}}
  \FunctionTok{mutate}\NormalTok{( }\hspace*{\fill}\NormalTok{\circled{4}}
    \AttributeTok{adj\_6th =} \FunctionTok{if\_else}\NormalTok{(}
\NormalTok{      math\_6th }\SpecialCharTok{\textless{}}\NormalTok{ state\_6th\_proficiency,}
\NormalTok{      (state\_6th\_proficiency }\SpecialCharTok{{-}}\NormalTok{ math\_6th }\SpecialCharTok{+} \DecValTok{1}\NormalTok{),}
      \DecValTok{0}
\NormalTok{    )}
\NormalTok{  ) }\SpecialCharTok{|\textgreater{}}
  \FunctionTok{mutate}\NormalTok{(}
    \AttributeTok{adj\_gpc =} \FunctionTok{if\_else}\NormalTok{(}
\NormalTok{      grad\_comp }\SpecialCharTok{\textless{}}\NormalTok{ state\_gpc,}
\NormalTok{      (state\_gpc }\SpecialCharTok{{-}}\NormalTok{ grad\_comp }\SpecialCharTok{+} \DecValTok{1}\NormalTok{),}
      \DecValTok{0}
\NormalTok{    )}
\NormalTok{  ) }\SpecialCharTok{|\textgreater{}}
  \FunctionTok{mutate}\NormalTok{(}
    \AttributeTok{adj\_academic =}\NormalTok{ (}
\NormalTok{      (((adj\_3rd) }\SpecialCharTok{+}\NormalTok{ (adj\_6th) }\SpecialCharTok{+}\NormalTok{ (adj\_gpc)) }\SpecialCharTok{/} \DecValTok{3}\NormalTok{)) }\SpecialCharTok{+} \DecValTok{1} \hspace*{\fill}\NormalTok{\circled{5}}
\NormalTok{  ) }\SpecialCharTok{|\textgreater{}}
  \FunctionTok{mutate}\NormalTok{(}
    \AttributeTok{scoScore =}\NormalTok{ (}
\NormalTok{      ((urm\_pct }\SpecialCharTok{*} \FloatTok{1.5}\NormalTok{) }\SpecialCharTok{+}\NormalTok{ (frl\_pct }\SpecialCharTok{*} \FloatTok{1.5}\NormalTok{) }\SpecialCharTok{+}\NormalTok{ (adj\_academic)) }\SpecialCharTok{/} \DecValTok{3} \hspace*{\fill}\NormalTok{\circled{6}}
\NormalTok{    )}
\NormalTok{  )}
\end{Highlighting}
\end{Shaded}

\begin{description}
\tightlist
\item[\circled{1}]
Take
\item[\circled{2}]
Then
\item[\circled{3}]
Then
\item[\circled{4}]
Repeat the process for each academic measure (3rd grade ELA, 6th grade
math, and graduation pathways completion rate)
\item[\circled{5}]
Calculate the Academic Score (\(S_a^\prime\))
\item[\circled{6}]
Calculate
\end{description}

The \texttt{echo:\ false} option disables the printing of code (only
output is displayed).

\hypertarget{school-corporation-opportunity-scores-for-indiana}{%
\subsection{School Corporation Opportunity Scores for
Indiana}\label{school-corporation-opportunity-scores-for-indiana}}

Quarto enables you to weave together content and executable code into a
finished document. To learn more about Quarto see
\url{https://quarto.org}.

{
\makeatletter
\def\LT@makecaption#1#2#3{%
  \noalign{\smash{\hbox{\kern\textwidth\rlap{\kern\marginparsep
  \parbox[t]{\marginparwidth}{%
    \footnotesize{%
      \vspace{(1.1\baselineskip)}
    #1{#2: }\ignorespaces #3}}}}}}%
    }
\makeatother

}

\hypertarget{understanding-relationships}{%
\subsubsection{Understanding
Relationships}\label{understanding-relationships}}

{
\makeatletter
\def\LT@makecaption#1#2#3{%
  \noalign{\smash{\hbox{\kern\textwidth\rlap{\kern\marginparsep
  \parbox[t]{\marginparwidth}{%
    \footnotesize{%
      \vspace{(1.1\baselineskip)}
    #1{#2: }\ignorespaces #3}}}}}}%
    }
\makeatother

}

\hypertarget{maps}{%
\subsubsection{Maps}\label{maps}}

{
\makeatletter
\def\LT@makecaption#1#2#3{%
  \noalign{\smash{\hbox{\kern\textwidth\rlap{\kern\marginparsep
  \parbox[t]{\marginparwidth}{%
    \footnotesize{%
      \vspace{(1.1\baselineskip)}
    #1{#2: }\ignorespaces #3}}}}}}%
    }
\makeatother

}



\end{document}
